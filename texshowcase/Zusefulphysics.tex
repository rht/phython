
\documentclass[hyperpdf,bindnopdf,twocolumn]{article}
\usepackage{
amsmath,
booktabs,
cancel,
caption,
cleveref,
colortbl,
csquotes,
datatool,
helvet,
mathpazo,
multirow,
listings,
pgfplots,
xcolor,}

%==============
%useful physics
\usepackage{siunitx}
\usepackage{physymb} % by David Zaslavsky
%==============



\begin{document}
    \section{Units}
    Numbers:\\
        \num{12345,67890} \\
        \num{1+-2i}       \\
        \num{.3e45}       \\
        \num{1.654 x 2.34 x 3.430}\\

    List/Range of numbers:\\
        \numlist{10;20;30}                    \\
        \SIlist{0.13;0.67;0.80}{\milli\metre} \\
        \numrange{10}{20}                     \\
        \SIrange{0.13}{0.67}{\milli\metre}\\

    Units:\\
        \si{kg.m.s^{-1}}                \\
        \si{\kilogram\metre\per\second} \\
        \si[per-mode=symbol]
        {\kilogram\metre\per\second}  \\
        \si[per-mode=symbol]
        {\kilogram\metre\per\ampere\per\second}\\
        \ang{10}\\ %degree

    Numbers and Units:\\
        \SI[mode=text]{1.23}{J.mol^{-1}.K^{-1}}\\

    SI base units:\\
    \si{\ampere}\\
    \si{\candela}\\
    \si{\kelvin}\\
    \si{\kilogram}\\
    \si{\meter }\\
    \si{\mole  }\\
    \si{\second}\\
    SI derived units:\\
    \si{\becquerel} \\
    \si{\newton}    \\
    \si{\degreeCelsius} \\
    \si{\ohm}       \\
    \si{\coulomb}   \\
    \si{\pascal}    \\
    \si{\farad}     \\
    \si{\radian}    \\
    \si{\gray}      \\
    \si{\siemens}   \\
    \si{\hertz}     \\
    \si{\sievert}   \\
    \si{\henry}     \\
    \si{\steradian} \\
    \si{\joule}     \\
    \si{\tesla}     \\
    \si{\katal}     \\
    \si{\volt}      \\
    \si{\lumen}     \\
    \si{\watt}      \\
    \si{\lux}       \\
    \si{\weber}     \\
    \si{\angstrom}     \\
    \si{\electronvolt}     \\


          
    \section{Formulas}

    From physymb:\\
    \begin{align*}
        \sn{6.23}{6}\\
        \snunit{3.1}{6}{meter^3}\\ % conflict with siunitx
        \asin \acos \sech \\
        \intset \realset \\
        \ud{y}{x}\\
        \pd{y}{x}\\
        \pddd{y}{x}\\
        \udc\\
        \iint e^{i\vec{k}\cdot\vec{x}}\udds\vec{x}\\ % use s because we want a small space between the integrand and the d symbol
        \div \grad \curl \lapl\\
        \conj{z}  \real{z} \realop \\
        \unitx \unity \uniti \unite \unitomega \unit{\phi}\\
        \commut{A}{B} \acommut{A}{B}  \herm{A}\\
        \commut{\frac{\hat{A}}{c}}{\hat{B}}\\
        \left[\hat{A},\hat{B}\right]\\
        \exd \hodge \\
        classmech \pbrac{f}{g} \\
        QM \ket{\phi} \braket{\phi}{\psi} \melement{\phi}{A}{\psi} \expect{A}
    \end{align*}
    

\end{document}
